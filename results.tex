\chapter{Results}
\label{cha:results}

\section{Candidate Event}
\label{sec:candidate-event}

\begin{figure}[!htbp]
  \centering
  \begin{subfigure}[b]{0.75\textwidth}
    \centering
    \includegraphics[width=\textwidth]{plots/event_display_eta_z.png}
    \caption{\label{fig:evt-eta-z}}
  \end{subfigure}
  \begin{subfigure}[b]{0.75\textwidth}
    \centering
    \includegraphics[width=\textwidth]{plots/event_display_phi_r.png}
    \caption{\label{fig:evt-phi-r}}
  \end{subfigure}
  \caption{Signal-like event candidate detected by the CMS. Shown are both the $\eta$-$z$- (\ref{fig:evt-eta-z}) and the $\phi$-$r$-plane (\ref{fig:evt-phi-r}).}
  \label{fig:evt-display}
\end{figure}

\noindent To summarize the event selection, an event that has been recorded by the CMS detector is being shown in figure~\ref{fig:evt-display}. Muons below the minimum transverse momentum of $20\,\text{GeV}$ and jets below the minimum energy of $30\,\text{GeV}$ have been removed. The event passed all selection requirements and visualizes the most important aspects of it. Both muons are highly energetic and well isolated, as they are reasonably separated from the various jets. The number of jets exceeds the minimum of two, while the respective jet energies also surpasses the threshold by quite a margin. With the low amount of missing transverse energy that is needed to pass the upper limit of $50\,\text{GeV}$, the red arrow representing this quantity is barely noticeable. The two muon and two jet candidates result in an invariant mass of $229.4\,\text{GeV}$ corresponding to the smuon $m_{\tilde{\mu}}$. Removing the leading muon from this invariant mass calculation yields $186.8\,\text{GeV}$, which is the gaugino mass $m_{\tilde{\chi}^0_1}$.

\section{Final Distribution}
\label{sec:final-dist}

Having accounted for the systematic uncertainties, one can examine the final distribution(s). This encompasses the mass of the gaugino and smuon. In case of the smuon, the selected two jets and the two muons enter the invariant mass calculation. Removing the leading muon from this composition yields the gaugino case. All event selection requirements are applied and in relation to the $b$-jet veto plus charge control region, only the inverted missing transverse energy requirement is reversed again $E_{\text{T}}^{\text{miss}} < 50\,\text{GeV}$. Figure~\ref{fig:final-dist} shows both distributions including the data-driven background estimate.

\begin{figure}[htb!]
  \centering
  \begin{subfigure}[b]{0.6\textwidth}
    \centering
    \includegraphics[width=\textwidth]{plots/m_gaugino.pdf}
    \caption{\label{fig:m_gaugino}}
  \end{subfigure}
  \begin{subfigure}[b]{0.6\textwidth}
    \centering
    \includegraphics[width=\textwidth]{plots/m_smuon.pdf}
    \caption{\label{fig:m_smuon}}
  \end{subfigure}
  \caption{Mass of the gaugino (\ref{fig:m_gaugino}) and smuon (\ref{fig:m_smuon}) after all selection requirements. Both are calculated from the invariant masses of the selected two jets and two muons in the smuon case, as well as two jets and the sub leading muon for the gaugino. The distributions include the data-driven background estimate.}
  \label{fig:final-dist}
\end{figure}

\noindent Table~\ref{tab:nev-msmuon} resolves each group of processes into its individual background components. The importance of the data-driven background estimate is shown by its large contribution. It is roughly twice as high as the biggest Monte Carlo sample contribution given by the $WZ \rightarrow 3l \nu$ process. Together, they amount to more than $80\,\pct$ of the entire background prediction.

\begin{table}[!htb]
  \centering
  \begin{tabular}{|l|d{4}|d{4}|d{4}|}
    \hline
    \multirow{2}{*}{Sample}    & \multirow{2}{0.6cm}{$N_{\text{Events}}$} & \multicolumn{2}{c|}{Uncertainties} \\ \cline{3-4}

                               &                                   & \text{stat.}  & \text{sys.}                       \\ \hline \hline
    Data-driven background estimate                   & 33.5                                 & 2.4    & 16.0                         \\ \hline \hline
    
    \multicolumn{4}{|c|}{Monte Carlo} \\ \hline    
    $W\gamma \rightarrow l\nu 2\mu$ & 1.35   & 0.45   & 0.69   \\
    $WZ \rightarrow 3l \nu$         & 17.1   & 0.5    & 1.9    \\
    $WZ \rightarrow 2q l \nu$       & 0.026  & 0.026  & 0.003  \\
    $ZZ \rightarrow 2l 2q$          & 0.044  & 0.030  & 0.0044 \\
    $ZZ \rightarrow 4l$             & 2.55   & 0.05   & 0.27   \\ \hline
    $t\bar{t} + W$                  & 1.00   & 0.16   & 0.30   \\
    $t\bar{t} + WW$                 & 0.0163 & 0.0019 & 0.0083  \\
    $t\bar{t} + Z$                  & 0.258  & 0.070  & 0.038  \\ \hline
    $WWW$                           & 0.842  & 0.081  & 0.085  \\
    $WWZ$                           & 0.186  & 0.034  & 0.02   \\
    $WZZ$                           & 0.186  & 0.013  & 0.001  \\
    $ZZZ$                           & 0.0554 & 0.0018 & 0.0005 \\ \hline
    $W^- W^-$                       & 1.38   & 0.17   & 0.7    \\
    $W^+ W^+$                       & 3.97   & 0.48   & 2.02   \\
    $WW$ Double-parton              & 0.24   & 0.07   & 0.12   \\ \hline
    $\sum$                          & 62.5   & 2.6    & 22.1   \\ \hline \hline
    Data                            & \multicolumn{1}{c|}{63}     & \multicolumn{1}{c|}{-}      & \multicolumn{1}{c|}{-}      \\ \hline
  \end{tabular}
  \caption{Detailed number of events for each background in the distributions at the final stage of the analysis.}
  \label{tab:nev-msmuon}
\end{table}

The statistical uncertainties for the Monte Carlo predictions stem from the generated number of events, which limit the accuracy of the prediction. While the fake estimate has been determined from data, the method itself inherits  statistical uncertainties from the background samples which are subtracted from the data when determining the tight-to-loose ratio. Adding the uncertainties for every bin of the single- and double-fake estimates in quadrature and propagating them according to equation~\eqref{eq:fakes}, yields the statistical uncertainty for this case. All systematic uncertainties are taken from the summary table (Tab.~\ref{tab:sys-uncertainties}). Individual cross section uncertainties are included in accordance to the table's description as well. Overall one can observe an excellent agreement of measurement and SM simulation.

\begin{figure}[ht!]
  \centering
  \begin{subfigure}[b]{0.6\textwidth}
    \centering
    \includegraphics[width=\textwidth]{plots/signal_1000_1200_m_smu_chi.pdf}
    \caption{\label{fig:sig-1000-1200-m-smu-chi}}
  \end{subfigure}
  \begin{subfigure}[b]{0.6\textwidth}
    \centering
    \includegraphics[width=\textwidth]{plots/signal_1000_200_m_smu_chi.pdf}
    \caption{\label{fig:sig-1000-200-m-smu-chi}}
  \end{subfigure}
  \caption{Two-dimensional smuon and gaugino mass distribution. Only the listed signal is shown. One can see that $m_0$ and $m_{1/2}$ determine which out of the six search regions SR 1-6 is the dominant one. The black lines represent the borders of the individual search regions.}
  \label{fig:sig-m-smu-chi}
\end{figure}

Taking the signal point with $m_0 = 1000\,\text{GeV}$ and $m_{1/2} = 200\,\text{GeV}$ in figure~\ref{fig:final-dist} as an example, one can see that while the distributions coincide in the gaugino mass distribution, they do not in the smuon one. A two-dimensional visualization of the two exemplary signal contributions in the gaugino-smuon mass distribution is shown in figure~\ref{fig:sig-m-smu-chi}. In general, the dominant search region for a RPV SUSY signal depends on the values of $m_0$ and $m_{1/2}$. To improve the sensitivity of the limit setting procedure, a multi-bin approach is being employed. For that purpose the two-dimensional gaugino-smuon mass distribution is divided into six regions SR 1-6. Figure~\ref{fig:m_smu_chi} shows the two-dimensional, final distribution of the smuon and gaugino mass.

\begin{figure}[!htb]
  \centering
  \includegraphics[width=0.7\textwidth]{plots/m_smu_chi.pdf}
  \caption{Background estimate and CMS data of the two-dimensional smuon and gaugino mass distribution. All backgrounds are summed up and are shown as coloured search regions, while the data points are shown individually. The six bins are regarded as separate search regions SR 1-6 to improve the results of this analysis. The black lines represent the borders of the individual search regions.}
  \label{fig:m_smu_chi}
\end{figure}

\noindent Most entries are centered around one diagonal line, as a opposed to the previously shown signal samples. The contents of each of the six regions are summarized in table~\ref{tab:m_smu_chi_summary}.

\begin{table}[!htbp]
  \centering
  \begin{tabular}{|l|d{5}@{}d{4}|d{4}@{}d{4}|d{4}@{}d{4}|}
    \hline
    Process group  & \multicolumn{1}{c}{\text{SR 1}} &              & \multicolumn{1}{c}{\text{SR 2}} &              & \multicolumn{1}{c}{\text{SR 3}} &              \\ \hline
    Fake Estimate  & 16.0                            & \pm\: 7.8    & 10.7                            & \pm\:  5.3   & 5.00                            & \pm\:  2.58  \\ 
    $tt+V$         & 0.47                            & \pm\: 0.16   & 0.49                            & \pm\:  0.17  & 0.187                           & \pm\:  0.089 \\ 
    $VV$           & 8.46                            & \pm\: 1.22   & 7.1                             & \pm\:  1.1   & 4.01                            & \pm\:  0.62  \\
    $VVV$          & 0.374                           & \pm\: 0.064  & 0.392                           & \pm\:  0.066 & 0.209                           & \pm\:  0.045 \\ 
    Rare           & 1.04                            & \pm\: 0.57   & 1.33                            & \pm\:  0.72  & 1.65                            & \pm\:  0.88  \\ \hline
    $\sum$         & 26.4                            & \pm\: 9.6    & 20.0                            & \pm\:  7.1   & 11.1                            & \pm\:  4.0   \\ \hline
    Data           & 19                              &              & 25                              &              & 13                              &              \\ \hline \hline
    Process group  & \multicolumn{1}{c}{\text{SR 4}} &              & \multicolumn{1}{c}{\text{SR 5}} &              & \multicolumn{1}{c}{\text{SR 6}} &              \\ \hline
    Fakes Estimate & 0.23                            & \pm\: 0.20   & 1.15                            & \pm\:  0.89  & 0.33                            & \pm\:  0.25  \\
    $tt+V$         & <0.001                          &              & 0.032                           & \pm\:  0.029 & 0.094                           & \pm\:  0.062 \\
    $VV$           & 0.106                           & \pm\: 0.034  & 1.07                            & \pm\:  0.16  & 0.277                           & \pm\:  0.062 \\
    $VVV$          & 0.0062                          & \pm\: 0.0048 & 0.090                           & \pm\:  0.027 & 0.028                           & \pm\:  0.014 \\
    Rare           & 0.102                           & \pm\: 0.089  & 0.65                            & \pm\:  0.38  & 0.80                            & \pm\:  0.46  \\ \hline
    $\sum$         & 0.45                            & \pm\: 0.25   & 3.0                             & \pm\:  1.3   & 1.5                             & \pm\:  0.7   \\ \hline
    Data           & 2                               &              & 1                               &              & 3                               &              \\ \hline
  \end{tabular}
  \caption{Summary of the six regions displayed in figure~\ref{fig:m_smu_chi}. They are numbered starting on the left and progressing upwards from the lowest bin. Statistical and systematic uncertainties are added in quadrature.}
  \label{tab:m_smu_chi_summary}
\end{table}

\section{Calculation of Limits}
\label{sec:calc-of-limits}

As any excess of data is too small to be considered statistically significant, the result is translated into limits onto two quantities. The cross section of a $R$-parity violating supersymmetry model and its model parameter $\lambda^\prime_{211}$. To calculate these, one method commonly used in the CMS experiment will be utilized. Details are given in the upcoming section.


\subsection{CLs Method}
\label{sec:cls-method}

The $CL_s$ method~\cite{cls,cls2} is a modified frequentist analysis. With frequentist methods, one may exclude signal contributions based on downward fluctuations of the measurement in cases with low statistics. The $CL_S$ method, however, yields meaningful results in these circumstances. Its general idea is to compare two hypotheses to a measurement.

For most analyses in high energy particle physics, there are two cases to be covered. The null hypotheses $H_0$ being the Standard Model prediction by itself and a combination of the proposed signal and the Standard Model prediction as the second hypothesis $H_1$. In the most simple case, one expects the number of events $n$ to follow a Poisson distribution.

\begin{equation}
  \label{eq:poisson-likelihood}
    \mathcal{L} (x; n) = \frac{x^n}{n!} e^{-x}
\end{equation}

\noindent These likelihoods $\mathcal{L}_x$ are the basis for this method. Here, $x$ denotes the background-only scenario for $x = b$ and the background plus signal one for $x = s + b$. The distributions are ultimately a function of the cross section of the relevant processes, as it is proportional to the number of selected events $n$.

Using the two likelihood functions, a test statistic $Q$ can be defined. It is called the \textit{likelihood-ratio} and is used to judge how well a hypothesis describes the measurement. In the simplest case, it is given by

\begin{equation}
  \label{eq:testq}
  Q = -2 \ln{ \frac{\mathcal{L} (s + b; n)}{\mathcal{L} (b; n)} }.
\end{equation}

\noindent Large values of $Q$ correspond to a better agreement with $H_1$, while small ones to a better agreement with $H_0$. For a result of an actual experiment, the observed number of events $n_{\text{obs}}$ yields a set likelihood-ratio $Q_{\text{obs}}$. The confidence level for an hypothesis is defined as the probability $P$ to find a test statistic $Q$, which is less or equal to $Q_{\text{obs}}$. 

\begin{equation}
  \label{eq:cl-prob}
  CL_x = P (Q \leq Q_{obs}).
\end{equation}

To determine this probability $P$, one has to integrate over all possible $Q$s up to $Q_{\text{obs}}$.

\begin{equation}
  \label{eq:clx}
  CL_x = \int^{Q_{\text{obs}}}_{-\infty} \frac{d f_x(Q)}{d Q} \text{d} Q
\end{equation}

\noindent The probability distribution functions are denoted by $f_x(Q)$ and are determined through pseudo-experiments. For a very well understood background hypothesis, a confidence level in the null hypothesis close to one $CL_b (Q_{\text{obs}}) \approx 1$ is necessary to claim a discovery. The specific thresholds for the claim are given on $1 - CL_b (Q_{\text{obs}})$ and have been derived from standard deviations of the Gaussian distribution. This means $2.7 \cdot 10^{-3}$ corresponds to $3\,\sigma$, which is considered ``evidence'', and $5.7 \cdot 10^{-5}$ corresponding to $5\,\sigma$, a discovery.

Excluding the signal (plus background) hypothesis uses the eponymous $CL_s$ quantity instead of $CL_{s+b}$. It is defined as a ratio:

\begin{equation}
  \label{eq:cls}
  CL_s = \frac{CL_{s+b}}{CL_b}.
\end{equation}

\noindent While $CL_{s+b}$ may lead to unphysical results in certain cases, $CL_s$ does not. A prime example is a signal hypothesis that is dominated by its background. Slight fluctuations towards lower values of the latter would lead to a very low $CL_{s+b}$, yielding a strong exclusion limit. As $CL_s$ is not a confidence limit, but a ratio of those, fluctuations like these cancel themselves out.

A signal can be excluded at a confidence level $CL$, when the following relation is fulfilled.

\begin{equation}
  \label{eq:cl-excl}
  1 - CL_s \leq CL
\end{equation}

\section{Modifications}
\label{sec:mods}

Following the recommendations by the CMS and ATLAS collaborations~\cite{clsmod}, a modified test statistic is employed for the $CL_s$ method. To scale the signal strength, a parameter $\mu$ is introduced to the formula. It is used to in- or decrease the cross section of the signal prediction while the branching ratios remain constant. Additionally, both the signal and background simulation are subject to a number of uncertainties. In this analysis, the systematic uncertainties discussed in chapter~\ref{cha:systematics} are considered. These include detector based uncertainties, such as the energy resolution of jets, theoretical uncertainties, for example the accuracy of cross section calculations, and physics object identification uncertainties, like the ability to accurately determine the flavour of a jet. Additionally, the accuracy of the Monte Carlo simulation, which is limited by its simulated number of events, is also taken into consideration. To account for those in the limit calculation, a set of nuisance parameters $\theta$ is introduced.

\begin{equation}
  \label{eq:q-mod}
  Q = - 2 \ln{ \frac{\mathcal{L} (\mu s + b; n, \hat{\theta})_\mu) }{ \mathcal{L} (\hat{\mu} \cdot s + b; n, \hat{\theta} )} } \quad \text{ with } \quad 0 \leq \hat{\mu} \leq \mu
\end{equation}

\noindent Here, the pair of parameters $\hat{\mu}$ and $\hat{\theta}$ are evaluated at the global maximum of the likelihood. On the other hand, $\hat{\theta}_\mu$ denotes the conditional maximum likelihood estimator of $\theta$, which also depends on the choice of $\mu$. The lower constraint $0 < \hat{\mu}$ ensures that the signal has a positive contribution\footnote{While certain signals can lead to a negative interference, these are not covered by the $CL_s$ method.}. To prevent upward fluctuations of the measurement $\hat{\mu} > \mu$ to be considered evidence \textit{against} the signal hypothesis, the upper constraint $\hat{\mu} < \mu$ is set.

% Using the logarithm with a factor of $-2$ allows the test statistic to converge against $\Delta \chi^2$ for a large number of selected events.

To improve the limits, the search region is subdivided into multiple search regions. When combining the results for the individual regions, the respective likelihoods have to be multiplied and the overall value is given by

\begin{equation}
  \label{eq:likelihood-product}
  \mathcal{L} = \prod_i \mathcal{L}_i.
\end{equation}

\noindent This enhances the sensitivity, as the influence of large deviations in one region is increased.

\section{Limit Graphs}
\label{sec:limit-graphs}

The actual calculation of limits is performed by the \textsc{HiggsCombine} tool~\cite{clsmod,higgscombine}. It employs the \textsc{RooStats} package~\cite{roostats} which is part of the statistical analysis framework \textsc{ROOT}. In line with most CMS publications, the exclusion ranges are calculated at a $95\,\pct$ confidence level as an upper limit on the cross section ($CL_s \leq 0.05$). The necessary scaling of the signal strength to reach this $CL$ can be achieved through varying the previously introduced modifier $\mu$. This has to be done for each of the individual points in the RPV SUSY phase space.

While the result of the calculation is a limit on the signal cross section $\sigma$, it can be translated into a limit onto the model parameter $\lambda^\prime_{211}$. As the former scales quadratically with the latter, the following formula has to be used.

\begin{equation}
  \label{eq:xs-lambda-scale}
  \sigma \propto \lambda^{\prime\:2}_{211} \Rightarrow \lambda^\prime_{211} (95\,\pct\: CL) = \lambda^\prime_{211} \cdot \sqrt{ \frac{\sigma (95\,\pct\: CL)}{\sigma} } 
\end{equation}

The respective $95\,\pct\: CL$ multi-bin upper cross section limits, which have been translated into limits on the model parameter $\lambda^{\prime}_{211}$ are displayed as a function of $m_0$ and $m_{1/2}$ in figure~\ref{fig:lambda-prime-limits} and as a function of $m_{\tilde{\mu}}$ and $m_{\tilde{\chi}^0}$ in this RPV model in figure~\ref{fig:smu-nt0-limits}. One should note that, while $m_{\tilde{\chi}^0}$ corresponds to the LSP right before its RPV decay, $m_{\tilde{\mu}}$ represents the initial supersymmetric particle. Consequently, the sneutrino case $m_{\tilde{\nu}}$ is also included. As explained in the signal study (Cha.~\ref{cha:sig}), the LSP might be the product of prior RPC decays from the smuon, instead of being a direct decay product. With this in mind, one has to be careful when interpreting the latter two limit graphs.

\begin{figure}[!htbp]
  \centering
  \begin{subfigure}[b]{0.85\textwidth}
    \centering
    \includegraphics[width=\textwidth]{plots/l211limits_MultiBin_expected_logz-colz.pdf}
    \caption{\label{fig:lambda-prime-exp}}
  \end{subfigure}
  \begin{subfigure}[b]{0.85\textwidth}
    \centering
    \includegraphics[width=\textwidth]{plots/l211limits_MultiBin_logz-colz.pdf}
    \caption{\label{fig:lambda-prime-obs}}
  \end{subfigure}
  \caption{Expected (\ref{fig:lambda-prime-exp}) and observed (\ref{fig:lambda-prime-obs}) $95\,\%\: CL$ upper limits on the $\lambda^{\prime}_{211}$ coupling. They are given as a function of $m_0$ and $m_{1/2}$, while the parameters $A_0 = 0\,\text{GeV}$, $\text{sgn}(\mu) = +1$ and $\tan{\beta} = 20$ are fixed. The white regions to the left are part of the $\tilde{\tau}$-LSP phase space, to which this analysis is not sensitive. The ones on the bottom lead to unphysical models, non-converging RGEs, no electroweak symmetry breaking or tachyonic solutions.}
  \label{fig:lambda-prime-limits}
\end{figure}

\begin{figure}[!htbp]
  \centering
  \begin{subfigure}[b]{0.90\textwidth}
    \centering
    \includegraphics[width=\textwidth]{plots/plot_smuon_neutralino_mass_expectedLimitsCombine.pdf}
    \caption{\label{fig:m-smu-nt0-exp}}
  \end{subfigure}
  \begin{subfigure}[b]{0.90\textwidth}
    \centering
    \includegraphics[width=\textwidth]{plots/plot_smuon_neutralino_mass_LimitsCombine.pdf}
    \caption{\label{fig:m-smu-nt0-obs}}
  \end{subfigure}
  \caption{Expected (\ref{fig:m-smu-nt0-exp}) and observed (\ref{fig:m-smu-nt0-obs}) $95\,\%\: CL$ upper limits on the $\lambda^{\prime}_{211}$ coupling. They are given as a function of $m_{\tilde{\mu}}$ and $m_{\tilde{\chi}^0}$, while the parameters $A_0 = 0\,\text{GeV}$, $\text{sgn}(\mu) = +1$ and $\tan{\beta} = 20$ are fixed.}
  \label{fig:smu-nt0-limits}
\end{figure}


\section{Discussion and Interpretation}
\label{sec:discussion}

Comparing the shown limits to the analyses of 2011 RPV SUSY by CMS (Fig.~\ref{fig:2011rpv}) and the D0 predecessor (Fig.~\ref{fig:auterrpv}), the phase space that is being covered has been expanded considerably. While the ATLAS collaboration has not published any comparable limits, their search for long-lived, heavy particles with a muon and a displaced vertex~\cite{atlasrpv} is sensitive to lower values of $\lambda^\prime_{2ij}$. With their analysis, they put upper limits on the lifetime of the neutralino, which is assumed to decay promptly in this search.

The excluded values of $\lambda^{\prime}_{211}$ are roughly a factor of 10 better than the ones presented by D0~\cite{auter,d0rpv}, and around a factor of 1.8 in regards to the 2011 analysis~\cite{2011rpv}. This marks these results as the world's best collider-based limits on the $R$-parity violating supersymmetry scenario with a single coupling dominance of the $\lambda^{\prime}_{211}$ parameter.

In general, the exclusion limits get progressively weaker towards higher values of $m_0$. As shown in figure~\ref{fig:susy-2dxs}, the cross section for the interesting processes decreases in this region. However, there is also a drop in selection efficiency for the high mass signal points. This is due to changes in the particle properties and composition of the decay chain. These differences have been mentioned and illustrated in the event selection (Cha.~\ref{cha:eventsel}), by providing three exemplary signal points as a comparison in the various distributions.

\begin{figure}[!p]
  \centering
  \begin{subfigure}[b]{0.495\textwidth}
    \centering
    \includegraphics[width=\textwidth]{plots/limit_m12_200.pdf}
    \setlength{\abovecaptionskip}{-20pt}
    \setlength{\belowcaptionskip}{10pt}
    \caption{$m_{1/2} = 200\,\text{GeV}$\label{fig:limit-m12-200}}
  \end{subfigure}
  \begin{subfigure}[b]{0.495\textwidth}
    \centering
    \includegraphics[width=\textwidth]{plots/limit_m12_500.pdf}
    \setlength{\abovecaptionskip}{-20pt}
    \setlength{\belowcaptionskip}{10pt}
    \caption{$m_{1/2} = 500\,\text{GeV}$\label{fig:limit-m12-500}}
  \end{subfigure}

  \begin{subfigure}[b]{0.495\textwidth}
    \centering
    \includegraphics[width=\textwidth]{plots/limit_m12_800.pdf}
    \setlength{\abovecaptionskip}{-20pt}
    \setlength{\belowcaptionskip}{10pt}
    \caption{$m_{1/2} = 800\,\text{GeV}$\label{fig:limit-m12-800}}
  \end{subfigure}
  \begin{subfigure}[b]{0.495\textwidth}
    \centering
    \includegraphics[width=\textwidth]{plots/limit_m12_1200.pdf}
    \setlength{\abovecaptionskip}{-20pt}
    \setlength{\belowcaptionskip}{10pt}
    \caption{$m_{1/2} = 1200\,\text{GeV}$\label{fig:limit-m12-1200}}
  \end{subfigure}

  \begin{subfigure}[b]{0.495\textwidth}
    \centering
    \includegraphics[width=\textwidth]{plots/limit_m12_1600.pdf}
    \setlength{\abovecaptionskip}{-20pt}
    \setlength{\belowcaptionskip}{10pt}
    \caption{$m_{1/2} = 1600\,\text{GeV}$\label{fig:limit-m12-1600}}
  \end{subfigure}
  \begin{subfigure}[b]{0.495\textwidth}
    \centering
    \includegraphics[width=\textwidth]{plots/limit_m12_2000.pdf}
    \setlength{\abovecaptionskip}{-20pt}
    \setlength{\belowcaptionskip}{10pt}
    \caption{$m_{1/2} = 2000\,\text{GeV}$\label{fig:limit-m12-2000}}
  \end{subfigure}

  \caption{One dimensional slices of the two-dimensional limits given in figure~\ref{fig:lambda-prime-limits} for fixed values of $m_{1/2}$. The one and two sigma bands include both statistical and systematic uncertainties.}
  \label{fig:1d-limits}
\end{figure}

By taking one dimensional slices out of the $m_0$-$m_{1/2}$ plane of the limits, one can see the evolution of the limits with regards to their uncertainties (Fig.~\ref{fig:1d-limits}). There are no rogue values deviating too much from the expectation, but certain structures can be observed. These are the product of the universal mass parameters leading to certain smuon and gaugino masses, corresponding to one or two of the search regions. Using the phase space point with $m_0 = 1000\,\text{GeV}, m_{1/2} = 1200\,\text{GeV}$ as an example, one can see in figure~\ref{fig:sig-1000-1200-m-smu-chi} that its distribution has its main contribution in the SR 5, where data and MC are in reasonable agreement. Hence, the observed and expected limit (Fig.~\ref{fig:limit-m12-1200}) are very close to each other. For $m_0 = 1000\,\text{GeV}, m_{1/2} = 200\,\text{GeV}$ on the other hand, the slight excess of data in the corresponding SR 4 leads to an observed limit that is a lot worse than the expectation.

\section{Conclusion and Outlook}
\label{sec:conclusion}

A search for resonant production of second generation sleptons via single coupling dominance of the $\lambda^{\prime}_{211}$ parameter has been presented. For this purpose, the full $\mathcal{L} = 19.7\,\text{fb}^{-1}$ of double muon events recorded by the CMS experiment at proton-proton collisions with a centre-of-mass energy of $\sqrt{s} = 8\,\text{TeV}$, is being used. The final state in question is composed of two well isolated, same-sign muons and at least two jets. An additional characteristic of the signature is a low amount of missing transverse energy, which is atypical for common SUSY events. As no significant excess of data has been observed in comparison to the Standard Model's prediction, the best limits to date have been set on the model parameter $\lambda^{\prime}_{211}$ down to values of $10^{-3}$. They cover a range of up to $2.5\,\text{TeV}$ for the universal mass parameter of scalar particles $m_0$ and up to $1.5\,\text{TeV}$ for the universal mass parameter of fermionic particles $m_{1/2}$. A complementary expression through the mass of the smuon $m_{\tilde{\mu}}$ and gaugino $m_{\tilde{\chi}^0}$ in this model has also been provided.

For future efforts in this particular search, an improved selection efficiency for the signal could be achieved through taking the varying signature distributions of different RPV SUSY phase space points into account. This would require either fully flexible or individually adjusted thresholds for specific regions of the $m_0$-$m_{1/2}$-phase space.

In addition, results allowing for a broader interpretation beyond the cMSSM using ``simplified models'', based on the knowledge gained about the most important decay channels, are also possible and being investigated.

\clearpage
\section*{Acknowledgements} 
\label{sec:acknowledgements}

I offer my sincerest gratitude to both my family and my institute. Amongst the many people who helped me along the way, there are some which I would like to highlight. First of all, Professor Thomas Hebbeker, who enabled me to develop my thesis at his institute. Martin Weber, who supervised me throughout most of my work and introduced me to my new best friend ``Emacs''. Sebastian Th\"uer, who vigorously endured my constant nagging and questioning. Arnd Meyer, the voice of wisdom in times of need. Lars Sonnenschein, who was always eager to push for results. Andreas G\"uth, miracle worker at the monstrosity called Monte Carlo generators. Daniel Teyssier, at efficiently working on efficiencies. My parents for allowing me to get this far and their omnipresent all around support. And everyone working at the institute, who all contribute to the work-oriented and yet relaxing atmosphere, allowing for an enjoyable experience throughout my time there.

%%% Local Variables: 
%%% mode: latex
%%% TeX-master: "document"
%%% End: 
